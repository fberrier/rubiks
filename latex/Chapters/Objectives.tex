% Chapter Template

\chapter{Objectives} % Main chapter title

\label{Chapter0} % Change X to a consecutive number; for referencing this chapter elsewhere, use \ref{ChapterX}

%----------------------------------------------------------------------------------------
%	SECTION 1
%----------------------------------------------------------------------------------------

\section{Learning Objectives}

Back when I studied Financial Mathematics, almost 2 decades ago, it was all about probability theory, stochastic calculus and asset (in particular derivatives) pricing. These skills were of course very sought after in the field of options trading, but were also often enough to get a job in algorithmic or systematic trading. By the middle of the 2010s, with the constant advances in computing power and storage, the better availability of off-the-shelves libraries and data sets, I witnessed a first revolution: the field of machine learning became more and more prominent and pretty much overshadowed other (more traditional maths) skills. More recently, a  second revolution has taken not only the world of finance, but that of pretty much every science and industry, by storm: we are now in the artificial intelligence age.
In 2019-2020, I decided it was time to see by myself what this was all about, and if the hype was justified. What better way to do that than embark on a proper MSc in Artificial Intelligence?
\\
Of all the modules I have studied over the last two years of the Royal Holloway MSc in AI, I have been the most impressed by DL and  NLP (itself arguably largely an application of DL) and particularly interested in AIPnT, especially our excursion in the field of graphs search (a very traditional CS topic, but which somehow I had not yet had a chance to study in much details). Even though I still believe there is a tremendous amount of malinvestment  everywhere, due in good part to the inability of the average investor to distinguish between serious and scammy AI applications and startups (the same obviously goes for blockchain applications, which might warrant another MSc?), I have totally changed my mind around the potential of DL, DRL and NLP and think they are incredibly promising. I have been astonished to see by myself, through several of the courseworks we have done during the MSc, how incredibly efficient sophisticated ML, DL and DLR algorithms can be, when applied well on the right problems. Sometimes they just vastly outperform more naive and traditional approaches to the point of rendering older approaches entirely obsolete (e.g CV, NLP, game solvers, etc...).
\\
For the project component of the MSc, I thought it would be interesting (and fun) for me to try and apply some of the DL, DRL and search techniques (from AIPnT) to a couple of single-player games, such as the sliding puzzle (of which some variations are well known under different names, e.g. the 8-puzzle and 15-puzzle) and of course the Rubiks' cube. I am in particular looking to solidify my understanding of DRL by  implementing and experimenting with concrete (though arguably of limited practical use) problems.

%----------------------------------------------------------------------------------------
%	SECTION 2
%----------------------------------------------------------------------------------------

\newpage
\section{Project's Objectives}

I am hoping with this project to implement and compare a few different methods to solve the SP and the RC. Both these puzzles have tremendously large state spaces (see section \hyperref[sec:Games]{Games} for details) and only one goal state. I am therefore likely to only succeed with reasonably small dimensional puzzles, especially since I have chosen for simplicity to implement things in Python.
Depending on the progress I will be able to make in the imparted time, I am hoping to try a mix of simple searches (depth first search, breadth first search, A* with simple admissible heuristics), then more advanced ones such as A* informed by heuristics learnt via DL and DRL, as well as try different architectures and network sizes and designs for the DL and DRL heuristics. Time permitting I would like to give a go at DQL, and maybe also compare things with some open-source domain-specific implementations (for instance a Kociemba Rubik's algorithm implementation, see e.g. \cite{Kociemba}).
\\
Along the way, I am also hoping to learn a bit about these two games that I have chosen to work on, and maybe make a couple of remarks on them that the reader of this thesis might find interesting.
