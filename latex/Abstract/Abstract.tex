
\noindent
\textbf{Motivation}
\label{Abstract}
\\
Reinforcement Learning (\textbf{RL}), exposed with brilliant clarity in \cite{Sutton1998}, has until recently known less success than we might have hoped for. Its framework is very appealing and intuitive. In particular, the mathematical beauty of Value iteration and Q iteration (\cite{WatkinsThesis}) for discrete state and action spaces, blindly iterating from \textit{any} initial value, is quite profound. Sadly, it had until recently proven hard to achieve practical success with these methods.
\\
Inspired however by the seminal success of Deep Mind's team in using Deep Reinforcement Learning (\textbf{DRL}) to play Atari games (\cite{Mnih2013})  and to master the game of Go (\cite{AlphaGo}), researchers have in recent years made a lot of progress towards designing algorithms capable of learning and solving, \textit{without human knowledge}, the Rubik's Cube (\textbf{RC}) - as well as similar single player puzzles - by using Deep Q-Learning (\textbf{DQL})  (\cite{DBLP:journals/corr/abs-1805-07470}) or search and (\textbf{DRL}) value iteration (\cite{https://doi.org/10.48550/arxiv.1805.07470}).
\\
In this project, I will attempt to implement a variety of solvers combining $A^{*}$ search and heuristics, some of which will be handcrafted, others which I will train on randomly generated sequences of puzzles using \textbf{DL}, \textbf{DRL} or \textbf{DQL}, to solve the 15-puzzle (and variations of different dimensions), as well as the Rubik's cube.
\\
\\
\textbf{Organisation of this thesis}
\\
In the first section, I will quickly describe what I hope to get out of this project, both in terms of personal learning, as well as in terms of tangible results (solving some puzzles!). In section 2, I will do a quick recap of the different methods that I will use in the project. Section 3 will be dedicated to discussing the mathematics of the sliding puzzle \textbf{SP} and the \textbf{RC}. I might throw a few random (but hopefully interesting) observations in there and give some references for the keen reader. I will then give an overview in section 4 of the code base I have developed - and put in the open on my github page (\cite{FB}) - to complete this project. Finally, in sections 6 and 7, I will present all my various results on respectively the sliding puzzle and the Rubik's cube before offering some conclusion in section 8.
\\
A couple of appendix sections complement the main text. The first one details how to use the code base, from constructing puzzles, to learning and solving them using all the different methods I have implemented for this project. The second describes a limitation I bumped into while using the kociemba library (a well-known open source 3x3x3 \textbf{RC} solver), and how I circumvented it.

