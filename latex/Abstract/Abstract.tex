
\label{Abstract}
\noindent \textbf{Motivation} -- Reinforcement Learning (\textbf{RL}), exposed with brilliant clarity in \cite{Sutton1998}, had until recently known less success than we might have hoped for. Its framework is very appealing and intuitive. In particular, the mathematical beauty of Value and Q iteration (\cite{WatkinsThesis}) for discrete state and action spaces, blindly iterating from \textit{any} initial value, is quite profound. Sadly, it had proven hard to achieve practical success with these methods. Inspired however by the seminal work of Deep Mind's team in using Deep Reinforcement Learning (\textbf{DRL}) to play Atari games (\cite{Mnih2013})  and to master the game of Go (\cite{AlphaGo}), researchers have made a lot of progress towards designing algorithms capable of learning and solving, \textit{without human knowledge}, the Rubik's Cube (\textbf{RC}) by using Deep Q-Learning (\textbf{DQL})  (\cite{DBLP:journals/corr/abs-1805-07470}) or search and (\textbf{DRL}) value iteration (\cite{https://doi.org/10.48550/arxiv.1805.07470}). I will attempt to implement a variety of solvers combining $A^{*}$ and heuristics, some handcrafted, others trained on sequences of puzzles using Deep Learning (\textbf{DL}), \textbf{DRL} or \textbf{DQL}, to solve the 15-puzzle and the Rubik's cube.
\\
\\
\textbf{Organisation of this thesis} -- In section 1, I discuss what I hope to get out of this project, both in terms of personal learning, as well as in terms of tangible results (solving some puzzles!). In section 2, I do a recap of the methods used in the project. Section 3 is dedicated to the mathematics of the sliding puzzle (\textbf{SP}) and the \textbf{RC}. I give then an overview of my implementation in section 4. Sections 5 and 6 present my results on respectively the \textbf{SP} and the \textbf{RC} and section 7 concludes. A couple of appendices complement the main text: the first one details how to use the code base, the second describes a limitation I bumped into while using the kociemba library.
\\
\\
\noindent \textbf{Acknowledgements} -- 
I am grateful to Bank of America for sponsoring my part time 2-year MSc in Artificial Intelligence at Royal Holloway and would like to thank my managers Mitrajit Dutta and Stephen Thompson for their support. The company of my colleague Saurabh Kumar, as he embarked on the same MSc journey and with whom I have had countless debates about the new cool stuff we learnt, has made the whole process much more enjoyable.
\\
I would also like to thank Professor Chris Watkins for supervising my project. It was an honour to attend his lectures on \textbf{DL} during the 2020-21 school year, and I tried to absorb some of his wisdom at our meetings during this project. I am glad to have followed his advice at our last meeting: I stopped shaving for a month, which did free enough time for me to implement \textbf{DQL} (joint policy and value learning), as well as an \textbf{MCTS} algorithm, to nicely complement my assortment of solvers.
\\
Finally, my wife Thanh Nh\~a is my biggest supporter in everything I do, and her continued encouragement in my pursuit of this MSc has made it so much easier to juggle between a demanding banking job, running 50 to 70 miles a week in preparation for the Chicago marathon, and the coursework, lectures and exam revisions.