% Chapter Template

\chapter{Objectives} % Main chapter title

\label{Chapter0} % Change X to a consecutive number; for referencing this chapter elsewhere, use \ref{ChapterX}

%----------------------------------------------------------------------------------------
%	SECTION 1
%----------------------------------------------------------------------------------------

\Section{Learning Objectives}

Back when I studied Financial Mathematics, almost 2 decades ago, it was all about probability theory, stochastic calculus and asset pricing. These skills were sought after in the field of derivatives trading, and often enough to get a job in algorithmic or systematic trading. By the middle of the 2010s, with the constant advances in computing power and storage, the better availability of off-the-shelves libraries and data sets, I witnessed a first revolution: machine learning (\textbf{ML}) became more prominent and started overshadowing more traditional maths skills. In recent years, a second revolution has taken not only the world of finance, but that of every science and industry, by storm: we are now in the artificial intelligence (\textbf{AI}) age. In 2019, I decided to see by myself what this was all about, and if the hype was justified. What better way to do that than embark on a proper MSc in Artificial Intelligence?
\\
\\
Of the modules I have studied over the last two years, I have been most impressed by \textbf{DL},  Natural Language Processing (\textbf{NLP}) and particularly interested in AI Principles \& Techniques (\textbf{AIPnT}), especially our excursion in graphs search. I have totally changed my mind around the potential of \textbf{DL}, \textbf{DRL} and \textbf{NLP} and think they are incredibly promising. I have been astonished to see by myself, through several of the MSc courseworks, how incredibly efficient sophisticated \textbf{ML}, \textbf{DL} and \textbf{DRL} algorithms can be, when applied well on the right problems, often vastly outperforming more naive and traditional approaches.
\\
\\
For the project component of the MSc, I thought it would be interesting (and fun) for me to apply some of the \textbf{DL}, \textbf{DRL} and search techniques from \textbf{AIPnT} to the sliding puzzle (\textbf{SP}) and of course the famous Rubik's cube (\textbf{RC}). I am in particular looking to solidify my understanding of \textbf{DRL} and \textbf{DQL} by  implementing and experimenting with concrete (though arguably of limited practical use) problems.

%----------------------------------------------------------------------------------------
%	SECTION 2
%----------------------------------------------------------------------------------------

\Section{Project's Objectives}

I am hoping to try a mix of simple searches such as depth first search (\textbf{DFS}), breadth first search (\textbf{BFS}) and A* with simple admissible heuristics, then move to more advanced ones such as A* informed by heuristics learnt via \textbf{DL} and \textbf{DRL}, as well as try different network architectures. Time permitting I would like to give a go at \textbf{DQL}, and maybe also compare things with some open-source domain-specific implementations (for instance a Kociemba Rubik's algorithm implementation, see e.g. \cite{Kociemba}).
\\
\\
Along the way, I am also hoping to learn a bit about the mathematics of the \textbf{SP} and the \textbf{RC}.
